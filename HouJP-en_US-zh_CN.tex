% !TEX TS-program = xelatex
% !TEX encoding = UTF-8 Unicode
% !Mode:: "TeX:UTF-8"

\documentclass{resume}
\usepackage{zh_CN-Adobefonts_external} % Simplified Chinese Support using external fonts (./fonts/zh_CN-Adobe/)
%\usepackage{zh_CN-Adobefonts_internal} % Simplified Chinese Support using system fonts
\usepackage{linespacing_fix} % disable extra space before next section
\usepackage{cite}
\usepackage[svgnames]{xcolor}
\usepackage{hyperref}
\hypersetup{colorlinks,breaklinks,
            urlcolor=[rgb]{0.2,0.2,0.6},
            linkcolor=[rgb]{0.2,0.2,0.6}
}

\begin{document}
\pagenumbering{gobble} % suppress displaying page number


\name{Ziyang Feng}
\centerline{Software Engineer: Backend Dev. \& Full Stack}
\vspace{2ex}
% {E-mail}{mobilephone}{homepage}
% be careful of _ in emaill address
% {E-mail}{mobilephone}
% keep the last empty braces!
\contactInfo{fuujiro@qq.com}{(+86) 155-2489-2259}{ \url{https://github.com/fuujiro}}

 
\section{\faGraduationCap\ Education}
\datedsubsection{\textbf{Waseda University}}{Sep. 2020 -- Expected Jul. 2022}
\textit{Master of Engineering} in Information, Production and Systems
\datedsubsection{\textbf{Dalian University of Technology}}{Sep. 2016 -- Jul. 2020}
\textit{Bachelor of Engineering} in Computer Science \& Technology

\section{\faUsers\ Project Experience}
\datedsubsection{\textbf{RPC Framework(\href{https://github.com/fuujiro/jiro-rpc-framework}{jiro-rpc-framework})}}{Sept. 2020 -- Oct. 2020}
%\role{Summer Intern}{Manager: xxx}
\begin{itemize}
  \item Through dynamic proxy, with the Netty transmission architecture, implemented automatic registration with asynchronous non-blocking and heartbeat notification, implemented a variety of serialization methods.
  \item Service registration and discovery based on Nacos realizes automatic service cancellation and load balancing strategies (random distribution, balanced weighted round-robin, consistent Hash).
%  \item Optimized xxx 5\%
\end{itemize}

\datedsubsection{\textbf{Distributed Systems (\href{https://github.com/fuujiro/MIT6.824-Labs}{MIT6.824-Labs})}}{May. 2020 -- Oct. 2020}
%\role{C, Python, Django, Linux}{Individual Projects, collaborated with xxx}
According to MapReduce and Raft papers, understand and reproduce the core framework of the paper.
%Brief introduction: xxx
\begin{itemize}
  \item Based on the MapReduce, implemented the master's distribution task and the worker's calculation task.
  \item Understand the term and heartbeat mechanism of the Raft protocol, and how to achieve strong consistency.
%  \item xxx
\end{itemize}

\section{\faSitemap\ Intern Experience}

\datedsubsection{\textbf{Tencent Games - LightSpeed \& Quantum Studios}}{Dec. 2020 -- Present}
%\role{\LaTeX, Maintainer}{Individual Projects}
%The goal of this task was to detect and correct spelling errors on Chinese essays.
\begin{itemize}
  \item With ZeroMQ, implemented an asynchronous and high-performance C/C++ service framework based on the Router-Dealer prototype, which provides asynchronous and concurrent support for game AI training.
  \item Use Python to develop integrated tools, such as automatically pulling, downloading, decompressing and matching the md5 value, and calculating the number of containers and automatically loading Unity games.
\end{itemize}

\datedsubsection{\textbf{Huawei Technologies Co., Ltd. - CloudBU}}{Jun. 2019 -- Aug. 2019}
%\role{\LaTeX, Maintainer}{Individual Projects}
%The goal of this task was to detect and correct spelling errors on Chinese essays.
\begin{itemize}
  \item Implemented the monitoring of the gateway by the middleware and realize the load balancing of the gateway.
  \item Based on lua and go to complete the internal components of the company, to realize the logic between sub-modules and improve the usability of the system.
\end{itemize}

% Reference Test
%\datedsubsection{\textbf{Paper Title\cite{zaharia2012resilient}}}{May. 2015}
%An xxx optimized for xxx\cite{verma2015large}
%\begin{itemize}
%  \item main contribution
%\end{itemize}

\section{\faCogs\ Skills}
\begin{itemize}[parsep=0.5ex]
  \item Skilled in Java, C++, Shell, Git and \LaTeX, Familiar with data structures \& algorithms and network programming, had good programming style.
  \item Understand the operating system's memory management, TCP/IP protocol stack, and IO multiplexing.
  \item Experienced in Spring, SpringBoot, MyBatis and other back-end frameworks, and basic design patterns.
  \item Experienced in Java concurrent programming, thread pool mechanism and the understanding of JUC library.
  \item Understand JVM memory distribution, class loader mechanism, garbage collection algorithm.
  \item Familiar with Linux command shell and Docker \& virtual environments, and CI/CD Tools.
\end{itemize}

\section{\faTrophy\ Academic Competitions}
\datedline{\textit{National University Student Innovation Project}~~Awarded in Robot-arm vision calibration}{Mar. 2019}
\datedline{\textit{\nth{1} Prize}~~Awarded in Liaoning Province University Student Computer Application Competition}{Dec. 2018}
\datedline{\textit{\nth{1} Prize}~~Awarded in National College Student Mathematics Competition Dalian Division}{Jul. 2017}
\datedline{\textit{\nth{1} Prize}~~Awarded in Dalian University of Technology "Ti" Cup Electronic Design Competition}{Jun. 2017}


\clearpage

\name{冯子扬 }
\centerline{求职意向: 后端开发 | 软件研发}
% {E-mail}{mobilephone}{homepage}
% be careful of _ in emaill address
\vspace{1ex}
\contactInfo{fuujiro@qq.com}{(+86) 155-2489-2259}{ \url{https://github.com/fuujiro/} }


% {E-mail}{mobilephone}
% keep the last empty braces!
%\contactInfo{xxx@yuanbin.me}{(+86) 131-221-87xxx}{}

\vspace{-1ex}
 
\section{\faGraduationCap\  教育背景}
\datedsubsection{\textbf{早稻田大学}~~ \ 硕士, 信息生产系统工学 }{2020.09 -- 预计2022.07毕业}

\datedsubsection{\textbf{大连理工大学}~~ \ 学士, 计算机科学与技术 }{2016.09 -- 2020.06}
\vspace{1ex}

%\vspace{-1ex}

\section{\faUsers\ 项目经历}
\datedsubsection{\textbf{RPC框架 (\href{https://github.com/fuujiro/jiro-rpc-framework}{jiro-rpc-framework})} }{2020.09 -- 2020.10}
%\role{实习}{经理: 高富帅}
%xxx后端开发
\begin{itemize}
  \item 通过动态代理,实现自动注册服务;实现了多种序列化方式;实现Netty传输和通用序列化接口。不断迭代项目的IO模型,从BIO到NIO(select和epoll),实现了异步非阻塞和心跳通知机制。
  \item 基于 Nacos 的服务注册与发现,实现自动注销服务和负载均衡策略(随机分发,平衡加权轮询,一致性Hash)。
\end{itemize}

\vspace{-1.5ex}

\datedsubsection{\textbf{MIT6.824分布式系统 (\href{https://github.com/fuujiro/MIT6.824-Labs}{MIT6.824-Labs})}}{2020.05 -- 2020.07}
%\role{Golang, Linux}{个人项目,和富帅糕合作开发}
\begin{onehalfspacing}
根据MapReduce,Raft等经典论文,理解并复现其论文核心框架。
\begin{itemize}
  \item 基于 MapReduce 算法, 实现了master的分发任务以及worker的计算任务,实现了一个word-counter。
  \item 理解Raft协议正常工作和心跳机制,如何竞选投票,如何实现强一致性。
\end{itemize}
\end{onehalfspacing}

% Reference Test
%\datedsubsection{\textbf{Paper Title\cite{zaharia2012resilient}}}{May. 2015}
%An xxx optimized for xxx\cite{verma2015large}
%\begin{itemize}
%  \item main contribution
%\end{itemize}

\section{\faSitemap\ 实习经历}
\datedsubsection{\textbf{腾讯游戏 - 光子工作室群~/~ 后台开发实习生}}{2020.12 -- Present}
\vspace{-0.5ex}
%\role{\LaTeX, Python}{个人项目}
\begin{onehalfspacing}
%优雅的 \LaTeX\ 简历模板, https://github.com/billryan/resume
\begin{itemize}
  \item 以ZeroMQ为网络通信组件,根据Router-Dealer原型来开发了一个异步高性能的C/C++后台服务框架,为游戏AI训练提供异步并发的支持,实现了动态接入服务节点和多种负载均衡策略。
  \item 使用Python开发工作室的集成工具类,如自动拉取下载解压缩并匹配md5值,处理腾讯云上的docker容器的创建销毁,以及计算容器数量归类和unity游戏自动加载等。
\end{itemize}
\end{onehalfspacing}

\datedsubsection{\textbf{华为技术有限公司 - 云核心网~/~软件开发实习生}}{2019.06 -- 2019.08}
\vspace{-0.5ex}
%\role{\LaTeX, Python}{个人项目}
\begin{onehalfspacing}
%优雅的 \LaTeX\ 简历模板, https://github.com/billryan/resume
\begin{itemize}
  \item 完成中间件对于网关流量的监控,实现对网关流量的负载均衡,理解限流算法的设计。
  \item 基于lua及c完成公司内部的组件和脚本测试,实现子模块分离,降低耦合性。
\end{itemize}
\end{onehalfspacing}

\vspace{-1.5ex}

\section{\faCogs\ 个人能力}
% increase linespacing [parsep=0.5ex]
\begin{itemize}[parsep=0.5ex]
  \item 熟悉 Java、C++、Python、Go,熟悉基本数据结构和算法,有良好的编程风格。
  \item 理解 ArrayList、HashMap、ConcurrentHashMap等Java类库的基本底层原理。
  \item 熟悉 Java并发编程,线程池机制以及JUC类库的使用。
  \item 使用 Spring、SpringBoot、MyBatis等后端框架,理解基本的设计模式。
  \item 理解 JVM内存分布,类加载机制,垃圾回收算法和常用GC调优策略。
  \item 熟悉 数据库的ACID原则和三大范式,MySQL的使用以及性能优化,Redis的RDB、AOF持久化机制以及缓存穿透和缓存雪崩。
  \item 理解操作系统的内存管理机制,熟悉TCP/IP协议栈,网络编程以及IO多路复用。
  \item 熟练使用 \LaTeX,MarkDown,Git等笔记及效率工具,熟悉Docker等沙盒虚拟环境,了解CI/CD等DevOps测试。
\end{itemize}

\vspace{-1ex}
\vspace{-1ex}

\section{\faTrophy\ 学术竞赛}
\datedline{\textit{国家级大学生创新项目}~机器人手臂视觉标定}{2019.03}
\datedline{\textit{一等奖}~辽宁省大学生计算机应用大赛}{2018.12}
\datedline{\textit{一等奖}~全国大学生数学竞赛大连赛区}{2017.06}
\datedline{\textit{一等奖}~大连理工大学“Ti”杯电子设计大赛}{2017.05}


%\section{\faInfo\ 其他}
%% increase linespacing [parsep=0.5ex]
%\begin{itemize}[parsep=0.5ex]
%  \item 技术博客: http://blog.yours.me
%  \item GitHub: https://github.com/username
%  \item 语言: 英语 - 熟练(TOEFL xxx)
%\end{itemize}

%% Reference
%\newpage
%\bibliographystyle{IEEETran}
%\bibliography{mycite}
\end{document}
