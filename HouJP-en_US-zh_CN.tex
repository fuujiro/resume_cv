% !TEX TS-program = xelatex
% !TEX encoding = UTF-8 Unicode
% !Mode:: "TeX:UTF-8"

\documentclass{resume}
\usepackage{zh_CN-Adobefonts_external} % Simplified Chinese Support using external fonts (./fonts/zh_CN-Adobe/)
%\usepackage{zh_CN-Adobefonts_internal} % Simplified Chinese Support using system fonts
\usepackage{linespacing_fix} % disable extra space before next section
\usepackage{cite}
\usepackage[colorlinks,linkcolor=blue]{hyperref}

\begin{document}
\pagenumbering{gobble} % suppress displaying page number


% \name{Ziyang Feng}
% \centerline{Software Development: Backend Dev. \& Full Stack}
% \vspace{2ex}
% % {E-mail}{mobilephone}{homepage}
% % be careful of _ in emaill address
% % {E-mail}{mobilephone}
% % keep the last empty braces!
% \contactInfo{fuujiro@qq.com}{(+86) 155-2489-2259}{ \url{https://github.com/fuujiro}}

 
% \section{\faGraduationCap\ Education}
% \datedsubsection{\textbf{Waseda University}}{Sep. 2020 -- Expected Jul. 2022}
% \textit{Master student} in Intelligent Computing 
% \datedsubsection{\textbf{Dalian University of Technology}}{Sep. 2016 -- Jul. 2020}
% \textit{Bachelor student} in Computer Science \& Technology

% \section{\faUsers\ Project Experience}
% \datedsubsection{\textbf{RPC Framework(\href{https://github.com/fuujiro/jiro-rpc-framework}{jiro-rpc-framework})}}{Sept. 2020 -- Oct. 2020}
% %\role{Summer Intern}{Manager: xxx}
% \begin{itemize}
%   \item Dynamic proxy; automatic registration; multiple serialization methods; asynchronous non-blocking and heartbeat notification mechanism.
%   \item Based on Nacos, implemented automatic service cancellation and load balancing strategy (LRU algorithm).
% %  \item Optimized xxx 5\%
% \end{itemize}

% \datedsubsection{\textbf{Distributed Systems (\href{https://github.com/fuujiro/MIT6.824-Labs}{MIT6.824-Labs})}}{May. 2020 -- Oct. 2020}
% %\role{C, Python, Django, Linux}{Individual Projects, collaborated with xxx}
% According to MapReduce and Raft papers, understand and reproduce the core framework of the paper.
% %Brief introduction: xxx
% \begin{itemize}
%   \item Based on the MapReduce, implemented the master's distribution task and the worker's calculation task.
%   \item Understand the normal working of the Raft protocol and the heartbeat mechanism, how to choose the master (delay to avoid have two master at the same time), and how to achieve consistency.
% %  \item xxx
% \end{itemize}

% \datedsubsection{\textbf{Time-Cross Applets}}{Dec. 2018 -- Jan. 2019}
% %\role{\LaTeX, Maintainer}{Individual Projects}
% The goal of this tool is to arrange the time for activities and improve office efficiency.
% \begin{itemize}
%   \item The front end implements Material Design components to realize the graphical animation effect.
%   \item The back-end uses the SSM framework, MySQL as the primary database, and Redis as the high-concurrency secondary database to implement Restful-style access requests. The md5 salt encryption is used for user information, and the public development interface is reserved.
% \end{itemize}

% \section{\faSitemap\ Intern Experience}

% \datedsubsection{\textbf{HUAWEI TECHNOLOGIES CO., LTD.}}{Jun. 2019 -- Aug. 2019}
% %\role{\LaTeX, Maintainer}{Individual Projects}
% %The goal of this task was to detect and correct spelling errors on Chinese essays.
% \begin{itemize}
%   \item Implemented the monitoring of the gateway by the middleware and realize the load balancing of the gateway.
%   \item Based on lua and go to complete the internal components of the company, to realize the logic between sub-modules and improve the usability of the system.
% \end{itemize}

% % Reference Test
% %\datedsubsection{\textbf{Paper Title\cite{zaharia2012resilient}}}{May. 2015}
% %An xxx optimized for xxx\cite{verma2015large}
% %\begin{itemize}
% %  \item main contribution
% %\end{itemize}

% \section{\faCogs\ Skills}
% \begin{itemize}[parsep=0.5ex]
%   \item Skilled in Java, C++, Shell, Git and \LaTeX, Familiar with data structures and algorithms and had good programming style.
%   \item Familiar with Spring, SpringBoot, MyBatis and other back-end frameworks, and basic design patterns.
%   \item Familiar with Docker and other sandbox virtual environments, and CI/CD Tools.
% \end{itemize}

% \section{\faTrophy\ Academic Competitions}
% \datedline{\textit{National University Student Innovation Project}~~Awarded in Robot-arm vision calibration}{Mar. 2019}
% \datedline{\textit{\nth{1} Place}~~Awarded in Liaoning Province University Student Computer Application Competition}{Dec. 2018}
% \datedline{\textit{\nth{1} Place}~~Awarded in National College Student Mathematics Competition Dalian Division}{Jul. 2017}
% \datedline{\textit{\nth{1} Prize}~~Awarded in Dalian University of Technology "Ti" Cup Electronic Design Competition}{Jun. 2017}


% \clearpage

\name{冯子扬 }
\centerline{求职意向: 后端开发 | 软件研发}
% {E-mail}{mobilephone}{homepage}
% be careful of _ in emaill address
\vspace{1ex}
\contactInfo{fuujiro@qq.com}{(+86) 155-2489-2259}{ \url{https://github.com/fuujiro/} }


% {E-mail}{mobilephone}
% keep the last empty braces!
%\contactInfo{xxx@yuanbin.me}{(+86) 131-221-87xxx}{}

\vspace{-1ex}
 
\section{\faGraduationCap\  教育背景}
\datedsubsection{\textbf{早稻田大学}~~ \ 硕士, Information, Production and Systems }{2020.09 -- 预计2022.07毕业}

\datedsubsection{\textbf{大连理工大学}~~ \ 学士, 计算机科学与技术 }{2016.09 -- 2020.06}
\vspace{1ex}

%\vspace{-1ex}

\section{\faUsers\ 项目经历}
\datedsubsection{\textbf{RPC框架 (\href{https://github.com/fuujiro/jiro-rpc-framework}{jiro-rpc-framework})} }{2020.09 -- 2020.10}
%\role{实习}{经理: 高富帅}
%xxx后端开发
\begin{itemize}
  \item 通过动态代理,实现自动注册服务;实现了多种序列化方式;实现Netty传输和通用序列化接口。不断迭代项目的IO模型,从BIO到NIO(select和epoll),实现了异步非阻塞和心跳通知机制。
  \item 基于 Nacos 的服务注册与发现,实现自动注销服务和负载均衡策略(随机分发,平衡加权轮询,一致性Hash)。
\end{itemize}

\vspace{-1.5ex}

\datedsubsection{\textbf{MIT6.824分布式系统 (\href{https://github.com/fuujiro/MIT6.824-Labs}{MIT6.824-Labs})}}{2020.05 -- 2020.07}
%\role{Golang, Linux}{个人项目,和富帅糕合作开发}
\begin{onehalfspacing}
根据MapReduce,Raft等经典论文,理解并复现其论文核心框架。
\begin{itemize}
  \item 基于 MapReduce 算法, 实现了master的分发任务以及worker的计算任务,实现了一个word-counter。
  \item 理解Raft协议正常工作和心跳机制,如何竞选投票,如何实现强一致性。
\end{itemize}
\end{onehalfspacing}

\vspace{-1.5ex}

\datedsubsection{\textbf{时间叉叉(小程序)}}{2018.12 -- 2019.01}
%\role{\LaTeX, Python}{个人项目}
\begin{onehalfspacing}
实现一个用于安排集体时间,提高办公效率的效率工具,《微信-SegmentFault小程序黑马赛》作品。
\begin{itemize}
  \item 前端实现Material Design组件,实现时间交叉区域的高亮动画,抓住用户眼球。
  \item 后端使用SSM框架,MySQL作为主数据库,实现Restful风格的请求,对于用户信息使用md5盐值加密,保留公共开发接口。
\end{itemize}
\end{onehalfspacing}

% Reference Test
%\datedsubsection{\textbf{Paper Title\cite{zaharia2012resilient}}}{May. 2015}
%An xxx optimized for xxx\cite{verma2015large}
%\begin{itemize}
%  \item main contribution
%\end{itemize}

\vspace{-1ex}

\section{\faSitemap\ 实习经历}
\datedsubsection{\textbf{华为技术有限公司~/~软件工程师}}{2019.06 -- 2019.08}
\vspace{-0.5ex}
%\role{\LaTeX, Python}{个人项目}
\begin{onehalfspacing}
%优雅的 \LaTeX\ 简历模板, https://github.com/billryan/resume
\begin{itemize}
  \item 完成中间件对于网关流量的监控,实现对网关流量的负载均衡,理解限流算法的设计。
  \item 基于lua及c完成公司内部的组件和脚本测试,实现子模块分离,降低耦合性。
\end{itemize}
\end{onehalfspacing}

\vspace{-1.5ex}

\section{\faCogs\ 个人能力}
% increase linespacing [parsep=0.5ex]
\begin{itemize}[parsep=0.5ex]
  \item 熟悉 Java、C++、Python、Go,熟悉基本数据结构和算法,有良好的编程风格。
  \item 理解 ArrayList、HashMap、ConcurrentHashMap等Java类库的基本底层原理。
  \item 熟悉 Java并发编程,线程池机制以及JUC类库的使用。
  \item 使用 Spring、SpringBoot、MyBatis等后端框架,理解基本的设计模式。
  \item 理解 JVM内存分布,类加载机制,垃圾回收算法和常用GC调优策略。
  \item 熟悉 数据库的ACID原则和三大范式,MySQL的使用以及性能优化,Redis的RDB、AOF持久化机制以及缓存穿透和缓存雪崩。
  \item 理解操作系统的内存管理机制,熟悉TCP/IP协议栈,网络编程以及IO多路复用。
  \item 熟练使用 \LaTeX,MarkDown,Git等笔记及效率工具,熟悉Docker等沙盒虚拟环境,了解CI/CD等DevOps测试。
\end{itemize}

\vspace{-1ex}
\vspace{-1ex}

\section{\faTrophy\ 学术竞赛}
\datedline{\textit{国家级大学生创新项目}~机器人手臂视觉标定}{2019.03}
\datedline{\textit{一等奖}~辽宁省大学生计算机应用大赛}{2018.12}
\datedline{\textit{一等奖}~全国大学生数学竞赛大连赛区}{2017.06}
\datedline{\textit{一等奖}~大连理工大学“Ti”杯电子设计大赛}{2017.05}


%\section{\faInfo\ 其他}
%% increase linespacing [parsep=0.5ex]
%\begin{itemize}[parsep=0.5ex]
%  \item 技术博客: http://blog.yours.me
%  \item GitHub: https://github.com/username
%  \item 语言: 英语 - 熟练(TOEFL xxx)
%\end{itemize}

%% Reference
%\newpage
%\bibliographystyle{IEEETran}
%\bibliography{mycite}
\end{document}
